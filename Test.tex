\documentclass[12pt, a4paper, openany]{report}
\usepackage[utf8]{inputenc}
\usepackage[T1]{fontenc}

% LANGUAGE
\usepackage[english]{babel}
\usepackage{enumitem}  % Enumerate improved

% MATH / Others
\usepackage{amsmath, amssymb, amsthm}  % Math symbols
\usepackage{physics}  % \norm and \abs
\usepackage{esvect, cancel}  % Misc., vectors, strikethrough
\usepackage{mhchem}  % Chemistry
\usepackage{siunitx}  % Units SI
\usepackage{minted}  % Code fences ([cache=false])
\setminted[python]{
    fontsize=\footnotesize,
    tabsize=4,
    rulecolor=black,
    xleftmargin=18pt,
    linenos,
    breaklines
}
\usemintedstyle{pastie}

\newtheorem{thm}{Theorem}
\newtheorem{lem}{Lemma}

\theoremstyle{definition}
\newtheorem{defn}{Definition}
\newtheorem*{prf}{Proof}

\theoremstyle{remark}
\newtheorem{rmk}{Remark}

% GEOMETRY
\usepackage[
    paper=a4paper,
    top=2.5cm,
    left=2.25cm,
    headheight=15pt,
    headsep=12pt,
    textwidth=16cm,
    textheight=24cm,
]{geometry}
\usepackage{parskip}  % Reformat paragraphs, no indent first line
\usepackage{enumitem}  % Enumerate improved
\usepackage{scrextend}  % Indent text with addmargin environment
\usepackage{graphicx}  % Include graphics
\graphicspath{{latex-img/}}
\usepackage{caption}  % Caption without figures


% HYPERLINKS
\usepackage{hyperref}
\hypersetup{
    colorlinks=true,
    linktoc=all,
    linkcolor=blue,
}

% HEADERS
\usepackage{fancyhdr}
    \pagestyle{fancy}
    \lhead{Lecture 1}
    \rhead{Samyak Shah}
    \renewcommand{\footrulewidth}{0.4pt}
    \renewcommand{\headrulewidth}{0.4pt}
\usepackage{etoolbox}  % Define chapter page style
    \patchcmd{\chapter}{\thispagestyle{plain}}{\thispagestyle{fancy}}{}{}

% TITLE PAGE
\title{Advanced Linear Algebra: Lecture 1}
\author{Samyak Shah}
\date{01/09/2023}

\begin{document}
% TITLE
\maketitle
% TOC
\pagenumbering{gobble}{
    \tableofcontents
    \thispagestyle{plain}
    \cleardoublepage
}
\pagenumbering{arabic}

\section{Fields}

\begin{defn}[\textbf{Field}]
    A \textit{field} is a triple $(F, +, \cdot)$ where $F$ is a set, and $+$ and $\cdot$ are binary operations on $F$
    (called \textit{addition} and \textit{multiplication} respectively) satisfying the following nine conditions (known as the \textit{field axioms})
    For all axioms, $a, b, c \in F$.
\end{defn}

\begin{enumerate}
    \item \textbf{Associativity of addition} Addition $(+)$ is an associative operation on $F$. i.e. $a + (b + c) = (a + b) + c$
    \item \textbf{Existence of additive identity} There is an identity element for addition. We know that this identity is unique, and we will denote it by $0$.
    \item \textbf{Existence of additive inverses} Every element $x$ of $F$ is invertible for $+$. We know that the additive inverse for $x$ is unique, and we will denote it by $-x$.
    \item \textbf{Commutativity of multiplication} Multiplication $(\cdot)$ is a commutative operation on $F$. i.e. $a\cdot b = b \cdot a$.
    \item \textbf{Associativity of multiplication} Multiplication is an associative operation on F. i.e \(a \cdot (b \cdot c) = (a \cdot b) \cdot c\)
    \item \textbf{Existence of multiplicative identity} There is an identity element for multiplication. We know that this identity is unique, and we will denote it by \(1\)
    \item \textbf{Existence of multiplicative inverses} Every element \(x\) of \(F\) except possibly \(0\) is invertible for \((\cdot)\). We know that the multiplicative inverse for \(x\) is unique, and we will denote it by \(x^{-1}\).
          We do not assume \(0\) is not invertible. We just do not assume that it is.
    \item \textbf{Distributive law} For all \(x, y, z\) in \(F\), \(x \cdot (y + x) = (x \cdot y) + (x \cdot z)\)
    \item \textbf{Zero-One Law} The additive identity and multiplicative identity are distinct; i.e. \(0 \neq 1\)
\end{enumerate}

Commutativity of addition can be proved from the other axioms, and is thus presented as a theorem.

\begin{thm}[\textbf{(Commutativity of addition)}]
    Let $F$ be any field. Then $+$ is a commutative operation on $F$.
\end{thm}

\begin{prf}
    Let $x,y \in F$. Since multiplication is commutative, we have
    $$(1 + x) \cdot (1 + y) = (1+y) \cdot (1+x)$$
    By the distributive law,
    $$((1 + x) \cdot 1) + ((1 + x) \cdot y) = ((1+y) \cdot 1) + ((1 + y) \cdot x)$$
    Since $1$ is the multiplicative identity,
    $$(1+x) + ((1+x)  \cdot y) = (1+y) + ((1 + y) \cdot x)$$
    and hence
    $$1 + (x + ((1 + x)  \cdot y)) = 1 + (y + ((1 + y) \cdot x))$$
    By the cancellation law for addition
    $$x + ((1 + x) \cdot y) = y ((1 + y) \cdot x)$$
    By commutativity of multiplication and the distributive law
    $$x + (y \cdot (1+x)) = y + (x \cdot (1 + y))$$
    and
    $$x + ((y \cdot 1) + (y \cdot x)) = y + ((x \cdot 1) + (x \cdot y))$$
    Since $1$ is the multiplicative identity and addition is associative
    $$x + (y + (y \cdot x)) = y + (x + (x \cdot y))$$
    and hence
    \[(x + y) + (y \cdot x) = (y + x) + (x \cdot y)  \]
    Since multiplication is commutative,
    \[ (x + y) + (x \cdot y) = (y + x) + (x \cdot y) \]
    and by the cancellation law for addition,
    \[ x + y = y + x \]
    Hence, \(+\) is commutative. \qed
\end{prf}

We can alternatively determine the definition of a field in terms of groups.

\begin{defn} [Group]
    A \textit{group} is a set \( G \) with an associative binary operation
    \[ \ast:  G \cross G \rightarrow G\ \]
    which means
    \[ (g, g') \mapsto g \ast g' \]
    Note that sometimes, the \( \ast \) will be omitted entirely, i.e. \( g \ast g' = g g' \) \\
    that has
    \begin{itemize}
        \item an identity \( e \) with \( e \ast g = g \ast e = g,  \forall  g \in G \)
        \item an inverse \( g^{-1} \) for each \( g \in G \), so \( g \ast g^{-1} = e = g^{-1} \ast g \)
    \end{itemize}
\end{defn}

\begin{defn}[Abelian Group]
    \( G \) is abelian, if \( \ast \) is commutative. i.e. \( g \ast h = h \ast g, \forall g, h \in  G \)

\end{defn}

\begin{defn}[Field]
    A field is an Abelian group \( (F, +) \) with additive identity \( 0 \) such that
    \begin{itemize}
        \item \( F^{*} = F \setminus \{0\} \) is an Abelian group \( (F^{*}, \cdot) \) and
        \item multiplication \( (\cdot) \) distributes over addition \( (+) \). i.e. \( a \cdot (b + c) = a \cdot b + a \cdot c \)
    \end{itemize}
\end{defn}

Note that if you are given an Abelian group \( (F, +) \) with additive identity \( 0 \), then just taking away the additive identity does not mean you now magically have a field. In other words, a field is not induced.
It only becomes a field if it has another group structure given to you.\\

If the field happens to be something like \( \mathbb{F}_p = \{ 0, 1, \dots, p-1 \} \), then the additive strucure does induce the multiplicative structure, but for most fields this isn't true. Everything you already know
\textbf{will} still apply to arbitrary vector fields, except for the notions of:
\begin{itemize}
    \item length \( \implies \) the notion of topology
    \item angle
    \item order \( (a=b) \)
\end{itemize}

\section{Vector Space}
\begin{defn}[Vector Space]
    A vector space over \( F \) is \( (V, +) \) Abelian group with an \textbf{action} of \( F \).
    \[ F \cross V \rightarrow V \]
    \[ (\alpha, v) \mapsto \alpha V \] that
    \begin{itemize}
        \item distributes over \( (+) \) on both sides. i.e. \( (\alpha + \alpha', v) \mapsto \alpha v + \alpha' v \) and \( (\alpha, v + v') \mapsto \alpha v + \alpha v' \)
        \item associative: \( \alpha (\beta v) = (\alpha \beta) v\). Note that these two multiplications are different. In the first case, it is a multiplication between the action and the field first, and then again between the action and the field. In the second case, it is a multiplication between the action and the action first, and then between the action and the field.
        \item \( 1v = v,  \forall  v \in  V \), where 1 is the multiplicative identity.
    \end{itemize}
\end{defn}

Note that typically the there are 9 axioms for a vector space and 8 for a field, and they are awfully similar looking. They can be better summarized by using the notion of groups as above.

\section{Homomorphism}
\begin{defn}[Homomorphism]
    A homomorphism of vector spaces is a linear map.
\end{defn}

Any map that respects the linear structure is known as a homomorphism. This is a term that you will see a lot in algebra courses.
i.e. If \( \mathcal{B} \) is a basis for \( V \), then function \( f \) mapping \( \mathcal{B} \rightarrow \mathcal{W} \) implies
that \( \exists \) a homomorphism \( \phi : V \rightarrow W \) with \( \phi|_\mathcal{B} = f \). The final symbol means that we are restricting
the homomorphism \( \phi \) to the basis \( \mathcal{B} \).

\begin{defn}[]
    \( \phi: V \rightarrow W \) has kernel \( \ker \phi = \{ v \in  V | \phi v = 0\} \), and image \( \mathrm{im} \phi = \{ \phi v | v \in  V \} \subseteq W \). These are both subspaces. This means that it is a vector space that is contained in the vector space that you've already been given.
\end{defn}

\begin{thm}[Rank-Nullity]
    \[ \dim ( \ker \phi ) + \dim ( \mathrm{im} \phi ) = \dim V \].
\end{thm}
The \( \dim ( \ker \phi) \) is the rank and the \( \dim (\mathrm{im} \phi) \) is the nullity.

\begin{defn}
    \( \phi \) is an isomorphism if \( \phi \) is \textbf{injective} and \textbf{surjective}.

\end{defn}


\end{document}